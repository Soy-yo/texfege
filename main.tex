% Cambiar por english si el TFG es en inglés
\documentclass[spanish,a4paper,12pt]{report}

\usepackage{texfege}

% Paquetes locales con comandos útiles (comentar si no se van a usar)
\usepackage{pkg/mathutils}

% ===========================
% Opciones de paquetes
% ===========================

% Nombre de los anexos en español
\addto\captionsspanish{
    \renewcommand{\appendixname}{Anexo}
    \renewcommand{\appendixtocname}{Anexos}
    \renewcommand{\appendixpagename}{Anexos}
}

% Para poner números en el texto y que se formateen automáticamente usar \num
% Por ejemplo, \num{12345.6} -> 12 345,6
\sisetup{%
    group-separator={\,},%
    group-minimum-digits={5},%
    output-decimal-marker={,}%
}

% Encabezado de las páginas (se muestra el capítulo actual a la izquierda)
\pagestyle{fancy}
\fancyhf{}
\fancyhead[L]{\leftmark}

% Archivo de bibliografía y macro para usar "et al." en vez de "y otros" cuando hay muchos autores
% (lo último sólo afectaría si el TFG está en español, en inglés se hace automáticamente)
\addbibresource{biblio.bib}
\xpatchbibmacro{name:andothers}{%
    \bibstring{andothers}%
}{%
    \textit{et al.}%
}{}{}

% Nombre de las secciones para cleveref
\addto\captionsspanish{
    \crefname{section}{sección}{secciones}
    \crefname{subsection}{sección}{secciones}
    \Crefname{section}{Sección}{Secciones}
    \Crefname{subsection}{Sección}{Secciones}
}


% ===========================
% Datos del TFG
% ===========================

% No eliminar ningún campo; si no se quiere usar se deja en blanco

\title{Título de mi Trabajo de Fin de Grado}
\titleenglish{Title of my Bachelor’s thesis, but this time in English}

\author{Nombre del Autor}
\tutor{Nombre del Tutor}

% Hay que definir así la fecha para que babel la traduzca
% Si se escribe a mano se puede eliminar \AtBeginDocument
\AtBeginDocument{\date{\today}}

\university{Universidad Complutense de Madrid}
\faculty{Facultad del Espacio}
\degree{Grado en Filología Marciana}
\course{2021/22}

% Valores válidos: ucm_simple, ucm_colored, ucm_black
% O añade tu logo a img/logos y escribe aquí el nombre del archivo
\logo{ucm_simple}

% Para cambiar el tamaño del logo si fuera necesario
% \renewcommand{\logowidth}{0.5}

\city{Madrid}

% Descomentar para cambiar el texto que se muestra para el autor/tutor
% \renewcommand{\authortext}{Autores}
% \renewcommand{\tutortext}{Directores}
    
% ===========================
% Inicio del documento
% ===========================
\begin{document}

\maketitle

\afterpage{%
    \null
    \thispagestyle{empty}%
    \renewcommand{\thepage}{0}
    \addtocounter{page}{-1}%
    \newpage
}

\pagenumbering{Roman}

% \begin{otherlanguage}{spanish}
\chapter*{\centering Resumen}
Resumen del TFG.

\section*{\centering Palabras clave}
Lista de palabras clave separadas por comas.
% \end{otherlanguage}

\blankpage

% Si el TFG es en inglés comentar esto y descomentar el anterior
\begin{otherlanguage}{english}
\chapter*{\centering Abstract}
Resumen del TFG en inglés.

\section*{\centering Keywords}
Lista de palabras clave separadas por comas en inglés.
\end{otherlanguage}

\blankpage

{
    % Comenta para usar la configuración de hipervínculos por defecto en el índice
    \hypersetup{linkcolor=black}
    \tableofcontents
}

\blankpage
\vfill\newpage
\pagenumbering{arabic}

\chapter{Introducción}\label{chap:chapter1}
Si el capítulo no es muy largo puedes escribirlo aquí directamente. Si no, puedes dividirlo en un archivo por sección e incluirlos con \texttt{\textbackslash input\{chapter1/archivo.tex\}} (o haz lo que quieras, es tu TFG) \cite{einstein}.
\chapter{Un capítulo}\label{chap:chapter2}
\chapter{Otro capítulo}\label{chap:chapter3}
\chapter{Último capítulo}\label{chap:chapter4}
\chapter{Conclusiones}\label{chap:chapter5}

\begin{appendices}
\chapter{Un apéndice}\label{appendix:appendix1}
\end{appendices}

\printbibliography[heading=bibintoc]

\end{document}
